\documentclass[12pt]{article}
\usepackage{listings}
\usepackage{graphicx}
\usepackage{minted}
\usepackage[T1]{fontenc}
\newlength{\drop}

\begin{document}
  \begin{titlepage}
    \centering
    \drop=0.1\textheight
    \vspace*{7\baselineskip}
    \rule{\textwidth}{1.6pt}\vspace*{-\baselineskip}\vspace*{2pt}
    \rule{\textwidth}{0.4pt}\\[\baselineskip]
    {\LARGE TeamTris}\\[0.2\baselineskip]
    \rule{\textwidth}{0.4pt}\vspace*{-\baselineskip}\vspace{3.2pt}
    \rule{\textwidth}{1.6pt}\\[\baselineskip]
    \scshape
    StartScreen Class Documentation \\
    West Lafayette, IN \\
    Feb 28th 2020\par
    \vspace*{2\baselineskip}
    Created by \\[\baselineskip]
    {\Large Steven Dellamore\par}
    {\itshape dellamoresteven@gmail.com\par}
    {\itshape TeamTris \\ CS407\par}
  \end{titlepage}
  
\tableofcontents
\newpage


\section{StartScreen}
\textbf{Author}: Steven Dellamore \\
\textbf{Description}: Startscreen will build the startscreen and create all the buttons needed for the user to get into a game with their friends. The mouseClicks and the keyboard imports all all forwarded to this class when gamestate == 0 \\



\subsection{constructor}
\textbf{Author}: Steven Dellamore 
\vspace*{1\baselineskip}
\begin{lstlisting}
constructor()
\end{lstlisting} 
\vspace*{1\baselineskip}
\textbf{Description}: The constructor gets called when making a startscreen object. It will init all the values and set up the socket listener for the server to send things too. Here are the init values of the class variables: 
\begin{minted}[fontsize=\footnotesize]{c}
this.TokenBoxText = ""; 
this.usernameBoxStroke = false; 
this.usernameText = "username"; 
this.usernameTextTouched = false; 
this.gameStateStartScreen = 0;  
this.titleAnimation = [300, 500, 400, 700];
\end{minted}
 These varibles will be updated throughout the life of start screen. \mintinline[fontsize=\footnotesize]{javascript}{this.TokenBoxText} will init the token box to nothing, since the user has yet to do anyhting. the \mintinline[fontsize=\footnotesize]{javascript}{this.usernameBoxStroke} will be set to false so the program knows if the user as tried to sumbit. \mintinline[fontsize=\footnotesize]{javascript}{this.titleAnimation = [300, 500, 400, 700];} is the starting position of the title, and will fall every X frames. \\


\textbf{\large{\\Parameters}}:\\
\textbf{void }: constructor takes no params\\\textbf{\large{\\Returns}}:\\\textbf{StartScreen }: An object of start class class

\subsection{draw}
\textbf{Author}: Steven Dellamore 
\vspace*{1\baselineskip}
\begin{lstlisting}
draw()
\end{lstlisting} 
\vspace*{1\baselineskip}
\textbf{Description}: This funcion will be ran at 60 frames a second and will call all the functions needed to draw the launch screen. \\


\textbf{\large{\\Parameters}}:\\
\textbf{void }: draw takes no arugments\\\textbf{\large{\\Returns}}:\\\textbf{void }: something should go ehre

\subsection{animateTitle}
\textbf{Author}: Steven Dellamore 
\vspace*{1\baselineskip}
\begin{lstlisting}
animateTitle()
\end{lstlisting} 
\vspace*{1\baselineskip}
\textbf{Description}: Will check and add/subtract the locations of the T's falling when you go to the launch screen. Once the animation is done, this function will return instantly. \\


\textbf{\large{\\Parameters}}:\\
\textbf{void }: animateTitle takes no arugments\\\textbf{\large{\\Returns}}:\\\textbf{void}

\subsection{drawUsernameBox}
\textbf{Author}: Steven Dellamore 
\vspace*{1\baselineskip}
\begin{lstlisting}
drawUsernameBox()
\end{lstlisting} 
\vspace*{1\baselineskip}
\textbf{Description}: This function will draw the username box onto the screen. It also checks to see if the user has already tried to go into a game without a username and will display the usernamebox stroke as red to indicate they need to fill it in. \\


\textbf{\large{\\Parameters}}:\\
\textbf{void }: drawUsernameBox takes no arugments\\\textbf{\large{\\Returns}}:\\\textbf{void}

\subsection{drawTitle}
\textbf{Author}: Steven Dellamore 
\vspace*{1\baselineskip}
\begin{lstlisting}
drawTitle()
\end{lstlisting} 
\vspace*{1\baselineskip}
\textbf{Description}: This function will draw the title (Teamtris) onto the launch screen. This function will also display the two T's falling in the word TeamTris. \\


\textbf{\large{\\Parameters}}:\\
\textbf{void }: drawTitle takes no arugments\\\textbf{\large{\\Returns}}:\\\textbf{void}

\subsection{drawTokenBox}
\textbf{Author}: Steven Dellamore 
\vspace*{1\baselineskip}
\begin{lstlisting}
drawTokenBox()
\end{lstlisting} 
\vspace*{1\baselineskip}
\textbf{Description}: This function will draw the token box once the user clicks "join game". It will display the token box and the accept button. Unlike other buttons, all mouse clicks are handled. \\


\textbf{\large{\\Parameters}}:\\
\textbf{void }: drawTokenBox takes no arugments\\\textbf{\large{\\Returns}}:\\\textbf{void}

\subsection{mouseClickedStart}
\textbf{Author}: Steven Dellamore 
\vspace*{1\baselineskip}
\begin{lstlisting}
mouseClickedStart()
\end{lstlisting} 
\vspace*{1\baselineskip}
\textbf{Description}: This function is being called whenever gamestate = 0 and the user clicks their mouse. It will check the current gamestate of the startScreen and check if they are clicking on different parts of the screen such as the create game or join game buttons. \\


\textbf{\large{\\Parameters}}:\\
\textbf{void }: mouseClickedStart takes no arugments\\\textbf{\large{\\Returns}}:\\\textbf{void}

\subsection{drawHighScoreButtonCheckMouse}
\textbf{Author}: Steven Dellamore 
\vspace*{1\baselineskip}
\begin{lstlisting}
drawHighScoreButtonCheckMouse()
\end{lstlisting} 
\vspace*{1\baselineskip}
\textbf{Description}: This function is being called whenever the user clicks with gamestate of the startscreen == 0. This function checks if the mouse is over the highscore button and returns true if it is, false if its not. \\


\textbf{\large{\\Parameters}}:\\
\textbf{void }: drawHighScoreButtonCheckMouse takes no arugments\\\textbf{\large{\\Returns}}:\\\textbf{bool }: true => If mouse is over score button false => If mouse is not over score button

\subsection{drawHighScoreButton}
\textbf{Author}: Steven Dellamore 
\vspace*{1\baselineskip}
\begin{lstlisting}
drawHighScoreButton()
\end{lstlisting} 
\vspace*{1\baselineskip}
\textbf{Description}: This function will draw the three bars in the bottom left of the screen. It will also check if the mouse is over the button and highlight according. \\


\textbf{\large{\\Parameters}}:\\
\textbf{void }: drawHighScoreButton takes no arugments\\\textbf{\large{\\Returns}}:\\\textbf{void}

\subsection{keyPressedStart}
\textbf{Author}: Steven Dellamore 
\vspace*{1\baselineskip}
\begin{lstlisting}
keyPressedStart()
\end{lstlisting} 
\vspace*{1\baselineskip}
\textbf{Description}: This function will be called whenever the user clicked on a button on the start screen. general/keyPressed.js is where this function will be called. \\


\textbf{\large{\\Parameters}}:\\
\textbf{void }: keyPressedStart takes no arugments\\\textbf{\large{\\Returns}}:\\\textbf{void}\end{document}